\documentclass[11pt,a4paper,roman]{moderncv}      
\usepackage[english]{babel}

\moderncvstyle{classic}                            
\moderncvcolor{black}                            

% character encoding
\usepackage[utf8]{inputenc}                     

% adjust the page margins
\usepackage[scale=0.73]{geometry}

% personal data
\name{Matz Andreas Haugen}{}
\email{matzhaugen@gmail.com}
\address{Valhallveien 13, 0196, Oslo, Norway}


\begin{document}

\recipient{To}{Editor, \\
               Journal of theoretical Biology,\\ 
               }
\date{\today}
\opening{\textbf{Initial submission of manuscript titled: ``Gompertz growth as infinitely correlated non-local growth''}}
\closing{Sincerely yours, \vspace{-1em}}


% \enclosure[Enclosed ]{
%               \\ List of two additional referees 
% %                  \hspace{0.0em} 5. GATE (2016) \hspace{2em} 9. Identity-Proof (Voter-Id)\\
% %                  2. Diploma (MarkSheet/Certificate)
% %                  \hspace{1.8em} 6. SLET (2021) \hspace{1.9em} 10. Resume\\        
% %                  3. B.Tech (MarkSheet/Certificate)
% %                  \hspace{2.3em} 7. UGC-NET (2022) \hspace{-0.1em} \\
% %                  4. M.Tech (MarkSheet/Certificate)
% %                  \hspace{2.1em} 8. Caste Certificate \\
%                  }
\makelettertitle



Respected Editor,
\\
%references such as what and how you got this information
\vspace{1em}
We hereby submit a manuscript where we show that the Logistic and the Gompertz models commonly used in population biology can be connected with a novel framework based on the microscopic domain. In doing this we connect the Gompertz model with a ubiquitous growth-activating field justified by physical and mathematical principles. This adds to the fundamental understanding sought for the phenomenology of the Gompertz model (see e.g. the work of Željko Bajzer). Our analysis is not entirely surprising as the literature is sprinkled with examples of Gompertz growth as a result of a ubiquitous field, including its first discovery. Nevertheless, we reinforce the current understanding with a mathematical framework and make explicit what perhaps has been implicit or unrecognized before. 

In creating this novel framework, we have also been able to include other common statistical functions in a unified understanding of growth processes, based on varying dependence in spacetime. These functions include the Gaussian, Exponential, and the Poisson.

We hope that you will consider this manuscript for publication in your journal as it seems fitting both from a mathematical and from a phenomenological standpoint. 

Thank you for your consideration.

\makeletterclosing
% \newpage


% Reviewers for journal of theoretical biology
% - Michael Levitt, Prof. at Stanford University michael.levitt@stanford.edu https://med.stanford.edu/profiles/Michael_Levitt
% - Marcin Molski Prof. of Chemistry at Adam Mickiewicz University mamolski@amu.edu.pl
% - Nigel Goldenfeld (nigel@uiuc.edu) https://guava.physics.uiuc.edu/~nigel/
% - Michael Victor Berry, U of Bristol, Prof. of Physics (asymptotico@bristol.ac.uk)
% - Shklovskii, B (shklo001@umn.edu) A.S. Fine Chair in Theoretical Physics; Professor , School of Physics and Astronomy
% - Steven Strogatz
% Extra reviewers:
% 1. Alain Barrat, CNRS & Aix-Marseille Université, France , Alain.Barrat@cpt.univ-mrs.fr
% 2. Daniel ben-Avraham, dbenavra@clarkson.edu, Professor of Physics, Clarkson University
% 3. Mark Lewis, marklewis@uvic.ca, Department of Mathematics and Statistics, University of Victoria
% 4. Alexandru Hening, Department of Mathematics at Texas A and M University,  ahening@tamu.edu
% 5. Pierre Magal, Institut de Mathématiques de Bordeaux, Université de Bordeaux, pierre.magal@u-bordeaux.fr
% 6. Mombach, J. C.,  Departmento de Física, Universidade Federal de Santa Maria, Santa Maria, 97105-900, Brazil, jcmombach@smail.ufsm.br or jcmombach@gmail.com

% \begin{enumerate}
% \item - Michael Levitt, Prof. at Stanford University michael.levitt@stanford.edu https://med.stanford.edu/profiles/Michael_Levitt
% \item R Schlickeiser, Institut für Theoretische Physik, Lehrstuhl IV: Weltraum- und Astrophysik, Ruhr-Universität Bochum, D-44780 Bochum, Germany, rsch@tp4.rub.de
% \end{enumerate}

% \textbf{List of Editorial board members}
% \begin{enumerate}
% \item Levine, Herbert
% \item Levin, Simon A.
% \item Marquet, Pablo A.
% \end{enumerate}

% \textbf{List of NAS members (in order of relevance)} 
% \begin{enumerate}
% \item Levine, Herbert
% \item Levin, Simon A.
% \item Michael Levitt 

% \item Nigel D. Goldenfeld 
% \item Evans, Steven N.
% \item Altshuler, Boris L.
% \end{enumerate}
% \textbf{List of possible referees}
% \begin{enumerate}
% % IOP REVIEWERS
% \item Martin Evans,Professor of Statistical Physics,  https://www2.ph.ed.ac.uk/~mevans/, m.evans@ed.ac.uk
% \item Prof Ralf Metzler, Institute of Physics & Astronomy
% University of Potsdam
% Karl-Liebknecht-Str 24/25, Haus 28
% D-14476 Potsdam-Golm 
% \item Michael Victor Berry, U of Bristol, Prof. of Physics, asymptotico(add at bristol.ac.uk) 
% \item Ronald Dickman, Professor of Physics, UFMG, dickman@fisica.ufmg.br
% \item Ewa Gudowska-Nowak, ewa.gudowska-nowak[at]uj.edu.pl , 
% Marian Smoluchowski
% Institute of Physics,
% Jagiellonian University
% \item Shamik Gupta Tata Institute of Fundamental Research, Mumbai, India, shamik.gupta-At-theory.tifr.res.in (Official)
% \item Alain Barrat, CNRS & Aix-Marseille Université, France , Alain.Barrat@cpt.univ-mrs.fr
% \item Daniel ben-Avraham, dbenavra@clarkson.edu, Professor of Physics, Clarkson University
% \item Tom Britton, Dean of Mathematics and Physics, Stockholm University, tom.britton@math.su.se
% \item Mark Lewis, marklewis@uvic.ca, Department of Mathematics and Statistics, University of Victoria
% \item Alexandru Hening, Department of Mathematics at Texas A and M University,  ahening@tamu.edu
% \item K-Y. Lam, Mathematics,  the Ohio State University, Columbus, lam.184@osu.edu 
% \item X. Zou, Western University, London, Ontario, xzou@uwo.ca
% \item S. Petrovskii, Mathematical Sciences, University of Leicester, sp237@le.ac.uk
% \item Pierre Magal, Institut de Mathématiques de Bordeaux, Université de Bordeaux, pierre.magal@u-bordeaux.fr 
% \item Vladimir N. Binhi, vnbin@mail.ru, Prokhorov General Physics Institute of the Russian Academy of Sciences, 38 Vavilov St., 119991 Moscow, Russia
% Is familiar with the biological effects due geomagnetic disturbances and fields.
% \item Andrei B. Rubin, rubin@biophysics.msu.ru, Faculty of Biology, Lomonosov Moscow State University, Leninskie Gory 1/12, 119234 Moscow, Russia
% Is familiar with the biological effects due geomagnetic disturbances and fields.
% \item Viacheslav V. Krylov, I.D. Papanin Institute for Biology of Inland Waters Russian Academy of Sciences, Borok, Russian Federation, krylovviacheslav@mail.ru
% Is intimately familiar with geomagnetic disturbances' effect on biology.
% \item Marcin Molski, mamolski@amu.edu.pl, Adam Mickiewicz University, physics, Gompertz and quantum systems. Has done work on connecting the gompertz model to quantum coherent systems.
% \item Francesco Piazza, Francesco.Piazza@cnrs-orleans.fr, Max Planck Institute for the Physics of Complex Systems, physics, SIR models.
% Has written one of the central papers modeling covid disease propagation from the SIR perspective. We have based our methods on their paper.
% \item Robert M May, robert.may@zoo.ox.ac.uk, Professor of Zoology, University of Oxford.
% \end{enumerate}


% Other possible reviewers:
% \begin{enumerate}
% \item Joel Miller
% \item Mark Chaplain
% \item Vishwesha Guttal
% \item Natalia Komorova
% \item Shigui Ruan
% \item Simon Levin

% \item Helen Wearing
% \item James Watmough
% \item Jane Heffernan
% \item Jing-an Cui
% \item Julien Arino

% \end{enumerate} 

\end{document}
