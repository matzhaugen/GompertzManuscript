\documentclass{article}
% \documentclass{sn-jnl}
% \documentclass[review]{elsarticle}

\usepackage{arxiv}
\usepackage{booktabs}
\usepackage{siunitx}
\usepackage[utf8]{inputenc} % allow utf-8 input
\usepackage[T1]{fontenc}    % use 8-bit T1 fonts
\usepackage{hyperref}       % hyperlinks
\usepackage{url}            % simple URL typesetting
\usepackage{booktabs}       % professional-quality tables
\usepackage{amsfonts}       % blackboard math symbols
\usepackage{nicefrac}       % compact symbols for 1/2, etc.
\usepackage{microtype}      % microtypography
\usepackage{lipsum}   % Can be removed after putting your text content
\usepackage{graphicx}
\usepackage[square,compress,numbers]{natbib}
\usepackage{doi}
\usepackage{amsmath}
\usepackage{float}
\usepackage{lineno}
\linenumbers

% \def\bibcommenthead{}%

\newcommand{\insertPngFig}[3]{
  \begin{figure}[H]
  \centering
  \includegraphics[width=8.4cm]{#1.png}
  \caption{#2}
  \label{fig:#1}
  \end{figure}
}


\newcommand{\insertPdfFig}[3]{
  \begin{figure}[H]
  \centering
  \includegraphics[width=8.4cm]{#1.pdf}
  \caption{#2}
  \label{fig:#1}
  \end{figure}
}

\newcommand{\insertSmallPdfFig}[3]{
  \begin{figure}[h]
  \centering
  \includegraphics[width=8.4cm]{#1.pdf}
  \caption{#2}
  \label{fig:#1}
  \end{figure}
}

% \title{On the nature of the Gompertz in relation to the epidemic in 2019-2021}
\title{Gompertz growth as infinitely correlated non-local growth}

% \graphicspath{{../figures/}}
%\date{September 9, 1985} % Here you can change the date presented in the paper title
%\date{}          % Or removing it

\author{Matz A. Haugen\footnote{hi}, \\
Independent Reseacher, Oslo\\
\texttt{matzhaugen@gmail.com} \\
  \And
  Dorothea Gilbert \\
  Independent Reseacher, Oslo\\
}


% Uncomment to remove the date
%\date{}

% Uncomment to override  the `A preprint' in the header
%\renewcommand{\headeright}{Technical Report}
%\renewcommand{\undertitle}{Technical Report}
\renewcommand{\shorttitle}{On the nature of the Gompertz}

%%% Add PDF metadata to help others organize their library
%%% Once the PDF is generated, you can check the metadata with
%%% $ pdfinfo template.pdf
\hypersetup{
pdftitle={A template for the arxiv style},
pdfsubject={q-bio.NC, q-bio.QM},
pdfauthor={},
pdfkeywords={},
}

\begin{document}
\maketitle

\begin{abstract}
Logistic and Gompertz growth has traditionally been connected by augmenting the logistic model with an extra parameter to obtain what is known as $\theta$-logistic growth or the Richards model for growth. In spite of this bridge, the biological foundation of Gompertz growth is only vaguely understood. 
To add to this biological foundation, we show that there is another way to connect the two growth modes by adding higher order terms in the growth equation, and present both a macroscopic and microscopic representation of growth where the latter involves a network of entities. 
The microscopic representation elucidates the source of Gompertz growth as the potential abundance and not the realized abundance, and an example of such a network is presented. 
For finite higher order terms we show that the growth equation can be represented by Gauss' Hypergeometric function, and only for infinitely many terms does the abundance lend itself conveniently to a log-transformation wherein the log-abundance growth rate decreases linearly, the trademark of Gompertz growth. We thus equate Gompertz growth with correlated microscopic entities that exhibit long-range or non-local interaction. 
This conclusion adds to the interpretation of biological systems that undergo Gompertz growth where long-range interaction could be interpreted as the source of growth being external to the individual entities, like a field that stimulates growth simultaneously across all entities. 
This field could be interpreted as an environmental stressor like air-pollution, age, radiation, etc, or an environmental stimulant like an electromagnetic field. 
Our new discovery allows for a unification of well-known growth models as follows: time independent and uncorrelated growth yields the exponential growth function, time dependent and uncorrelated growth yields Gaussian growth, time independent and pairwise correlated growth yields the logistic function and time independent and infinitely correlated growth yields the Gompertz function. 

\end{abstract}

% keywords can be removed
\keywords{Gompertz \and Coherence \and Network Analysis \and Self-organization}

\section*{Highlights}
\begin{itemize}
\item Logistic and Gompertz growth is connected with a novel framework based on microscopic principles.
\item Gompertz growth is connected with ubiquitous growth-activating field justified by physical and mathematical principles.
\item Growth processes is put in a larger contextual framework based on varying dependence in spacetime, including the Exponential, Gaussian and Poisson distributions. 
\end{itemize}

\section{Introduction}

In 1825, Gompertz published a paper on the nature of mortality where he discovered that from adulthood and onward ``the power to avoid destruction'' decreases exponentially with age. Equivalently, mortality rates increase in geometric progression for arithmetic increases in age, an observation now commonly referred to as Gompertz' law of mortality \cite{Gompertz1825}. Makeham (1889) later provided an interpretation to Gompertz' law where the recuperative force of a human, or ``vital force'', becomes less efficient with time \cite{makeham1889further}. Gompertz' law has later been known as the Gompertz model or Gompertz growth. 

\subsection{Gompertz' law in existing work}
This simple law has been shown to generalize to situations where mortality rates are substituted by many types of abundance variables, like the growth of animals, plants, bacteria, and cancer tumors \cite{weymouth1931relative,weymouth1931age,laird1964dynamics,zwietering1990modeling,skinner1994mathematical,starck1998avian,aggrey2002comparison,paine2012fit,benzekry2014classical,halmi2014evaluation,tjorve2010shapes}, exposing Gompertz growth as nearly ubiquitous in biological growth processes. The idea that the Gompertz' law can be applied to other biological systems was first proposed by Weymouth (1931) \cite{weymouth1931relative,weymouth1931age} and Winsor (1932) \cite{Winsor1932}. Gompertz' law has also been shown to apply to societal phenomena like growth in railway traffic, and the demand for goods and services, and sales of tobacco \cite{olshansky1997ever,prescott1922law,peabody1924growth}.

Alongside the many contexts wherein Gompertz' law has been observed, an almost equal number of efforts have been made to interpret and derive this law from various phenomenological principles \cite{bajzer1997mathematical}. This effort has often been accompanied by a phenomenological comparison of another popular growth model called logistic growth, or Verhulst growth, where ``the power to avoid destruction'' decreases linearly with time, as opposed to decreasing exponentially with time as seen with the Gompertz model. In a study comparing the Gompertz and the logistic growth models, Petroni et al. (2020) \cite{petroni2020logistic} show that Gompertz growth can be interpreted as \textit{maximally coherent} growth based on long-range interaction or correlation, while logistic growth is minimally coherent with non-collaborating elementary entities. With this interpretation, one is lead to postulate that the Gompertz model is a result of an external stimulus with perceived long-range interactions while the logistic model is a result of an internal stimulus with local interaction. As such, the logistic model has been popular both for communicable disease models \cite{harko2014exact,kroger2020analytical,schlickeiser2021analytical,heng2020approximately} and and for predator-prey theory \cite{berryman1992orgins}, where disease or injury spreads through pairwise interactions. 

A number of studies in the biological context have argued that Gompertz growth could be activated from an external factor which is ubiquitously and instantaneously introduced to the system, e.g. as a toxic agent or an environmental stressor to which the system gradually adapts \cite{neafsey1988gompertz,neafsey1989gompertz,thompson1990biphasic}. In the original case of mortality rates as a function of age, this interpretation would imply that time itself is the environmental ubiquitous stressor. Other environmental stressors like chronic radiation exposure have been observed to shift the Gompertz curve, an observation also theoretically developed by Sacher (1956) (\cite{sacher1956statistical}). Sacher modeled the toxic stressor as a stochastic perturbation to the physiologic state of a population wherein death of an entity would occur beyond some perturbation threshold. The interpretation of Gompertz growth as the result of a ubiquitous stochastic perturbation was recently revived by De Lauro et al. (2014) \cite{de2014stochastic} who arrived at the same result without alluding to a threshold value beyond which death was inevitable, but rather by letting the stochastic perturbation be the source of growth.

Another framework used to show how the Gompertz model can result from an external or ubiquitous stressor is chemical reaction theory. In reaction theory, a set of reactants combine to produce products, possibly in the presence of a catalyst. Morkov (2019) \cite{markov2019reaction} showed that the Gompertz model can be interpreted as growth under a catalyst whose effect diminishes at a rate independent of the growth rate of the reactant. In other words, the catalyst can be considered as an external and ubiquitous stimulant or stressor to the reactants, perfectly in line with the ubiquitous stochastic perturbation model from Sacher and De Lauro et al. On the other hand, if the logistic growth equation is recast into the reaction network equivalent, the catalyst is diminished by the growth of the reactant, which suggests that the source of growth is internal or at least dependent on the internal state. So also in reaction theory the Gompertz model could be seen as a result of a ubiquitous field while the logistic model could be seen as a result of internal interactions of the reactants. 

It is worth noting that the notion that Gompertz growth is maximally coherent and non-local has also been derived from quantum mechanical principles \cite{molski2003coherent}. Molski and Konarski (2003) showed that the Gompertz equation is the solution of a form of the time-independent Schrodinger equation. From this result, they derive the quantum coherent states which are non-local in space while moving along a classical time trajectory. This adds to our exhibits that the Gompertz model can be interpreted as non-local in origin. Coherent quantum states was originally formulated in the context of electromagnetic fields by \citet{glauber1963coherent} (for which he received the Nobel Prize). Perhaps some biological macroscopic growth phenomena are also due to such fields. For example, the role of internal electrical fields in regenerative growth is well established and is based on negative electrical potentials that act on sensitive cells \cite{becker1984electromagnetic}. 

In fact, it has been shown that cancer malignancies are mostly \textit{negative} in electric polarity with respect to the rest of the body, while nonmalignant (or benign) tumor pathology is mostly \textit{positive} in polarity \cite{langman1949technique}. Gompertz growth has also been widely observed in cancerous cells \cite{kozusko2003combining,laird1964dynamics}. One is then led to the interpretation that certain biological growth processes that exhibit Gompertz growth are coupled with an electromagnetic field of specific character. Consequently and unsurprisingly, the artificial disruption of such a field has been found to disrupt the mitotic cell cycle \cite{goodman1983pulsing,liboff1981alternating,norton1984bioelectric,liboff1984time}. 

On the topic of disease modelling, the Gompertz model has been shown by Wang et al. \cite{wang2012richards} to be excluded from a subset of models in the communicable disease paradigm. They used the common Susceptible-Infected-Recovered (SIR) model by \citet{kermack1927contribution} augmented by the Richards parameter and showed that the allowed parameter range with the communicable disease assumption does not cover the parameter value taken in the Gompertz scenario. If one incorporates a social proximity network on which the disease spreads, \citet{zonta2022virus} observed through computer simulations that a near Gompertz growth can occur but only for scale-free networks, a type of network that has been shown to be rare in human social networks \cite{clauset2009power,broido2019scale,holme2019rare}. 
We will show that one can also obtain a Gompertz curve from a ubiquitous stressor scenario based on microscopic interactions.
% Given these observations, any mortality pattern that shows a Gompertz characteristic seems not be a result of a communicable disease, but rather from a ubiquitous stressor. 
While many studies show the SIR model (and its extensions) as the fitted model in epidemics \cite{carletti2020covid,cooper2020sir,postnikov2020estimation,munoz2021sir,cooper2022dynamical,saikia2021covid}, some recent studies have successfully fitted a Gompertz model \cite{Ohnishi2020,Rypdal2020,Catala2020,rodrigues2020monte,Levitt2020}.

\subsection{Main Results and Roadmap}
In this paper, our primary aim is to investigate the source of growth in self-organized processes. Second, we will elaborate on the idea that the logistic model results from local interactions while the Gompertz model results from a field or equivalently from an infinite set of higher order interactions between the microscopic entities involved in the growth. We will devise a slightly different parametrization than the Richards model, the generalized Gompertz-Logistic equation which is commonly used to compare the two models (e.g. by \citet{petroni2020logistic}, \citet{tjorve2017use}, and \citet{wang2012richards}). In particular, instead of considering a decimal number as the unifying exponent we will appeal to integer exponents with a number of terms that tend towards infinity as we approach the Gompertz model. As we shall see, this is rooted in the microscopic interpretation of the growth system, where each subgroup of $n$ elementary entities interacts with all other equally sized subgroups each composed of mutually exclusive elementary entities. If these subgroups are present at all orders of cardinality, the Gompertz model emerges, while only pairwise interactions yield the logistic model. As seen in the aforementioned references, this is in line both with experimental and theoretical results. In presenting this argument, we present a novel bridge between the logistic and the Gompertz model which offers both a mathematical and biological foundation to the source of growth. Lastly, we will present a unifying framework in the growth context for the common functions Gaussian, Exponential, Logistic, Gompertz and Poisson.

Our paper is structured as follows: we will first present a new macroscopic foundation of the Gompertz-Logistic spectrum which will offer an alternative to the traditional Richards model. Then, we present a microscopic foundation which acts as the building blocks of the macroscopic phenomena. This is where we show that the Gompertz model results from an infinite set of higher order interactions between the elementary entities involved in the growth, which could alternatively be seen as a ubiquitous field or in the biological context: an environmental stressor. We proceed with a discussion that summarizes our findings in the context of the other well-known growth modalities involving the Gaussian and the Laplacian model and then conclude.

% \begin{enumerate}


%   \item Introduce gompertz and logistic growth
  
%   \item Discuss usages in litterature, cancer, mortality curves
%   \item Mention Bajzer 2020's effort to derive macroscopic foundation for Gompertz growth, but leaving out a microscopic foundation.
%   \item Reaction theory formulates gompertz as the interaction between growth variable $x$ and environmental ``catalyst'' $s$, where s is exponentially decreasing in size exactly analogous with an external source of growth. See https://biomath.math.bas.bg/biomath/index.php/biomath/article/view/j.biomath.2019.04.167/pdf
%   \item Reaction theory could also be self contained if we let s to s1, s2, .. going to infinity. Since this would imply long range correlation, the only explenation within our current knowledge of nature is to posit a field.
%   \item Molski interprets spatial gompertz behavior as a result of an EM field in space for molecular dynamics. The temporal equivalent would be to interpret the source of growth to be stressor field in time, equal across space.
%   \item De Lauro discusses Gompertz growth as a stochastic phenomena where the source of growth is a random and ubiquitous perturbation which aligns with the source of growth as external.
%   \item Patroni et al. (2020) \cite{petroni2020gompertz} show that gompertz growth can be interpreted as maximally coherent growth based on long-range cooperation or sharing of resources. In the extreme case when the long-range behavior is operating between all nodes, growth can be described as due to an external source inducing identical behavior across all nodes, which manifests itself as coherence.
%   \item Gompertz himself considered time as the ubiquitous external stressor in the original formulation of the Gompertz model.
%   \item Wang et al. (2012) \cite{wang2012richards} show that gompertz growth cannot emerge from intrinsic interactions only in a generalized logistic growth using the richards parametrization and a communicable disease paradigm.
%   \item The crux of our argument lies in showing that gompertz can be interpreted as an infinitely interacting growth equation, implying that growth is either communicating instantaneously, or triggered by an external source. And since the former is as odds with our current physical understanding but also perhaps overlapping with the second interpretation, we claim that the latter is the ostensible root of Gompertz growth. 
% \end{enumerate}

\section{Macroscopic growth}
\label{sec:macro}

First consider a system that grows according to logistic growth. If we let the variable $X(t)$ denote the abundance of a quantity in a normalized system where its maximum size is 1, the equation that determines its size as a function of time is
\begin{equation}
  \frac{1}{X}\frac{d}{dt} X(t) = \beta (1 - X(t)),
\end{equation}
where $\beta>0$ is a growth parameter. Here, the right hand side is commonly interpreted as representing ``resources available for growth'' which decrease linearly with respect to the relative abundance's growth rate on the left hand side. One way to generalize logistic growth is to introduce a parameter that exponentiates the resource availability in the following way,
\begin{equation}
\label{eq:Rich}
  \frac{1}{X}\frac{d}{dt} X(t) = \beta (1 - X^{\theta}(t)),
\end{equation}
also called $\theta$-logistic growth or Richards model \citep{richards1959flexible}. Petroni et al. (2020) argued that this exponentiation can be interpreted as non-linear interaction effects where there is a level of cooperation in the microscopic domain. They showed that in the limit where the system becomes \textit{maximally cooperative} or \textit{coherent}, $\theta\rightarrow 0$ and the $\theta$-logistic equation reduces in this limit to the Gompertz growth model,
 \begin{equation}
 \label{eq:Gomp}
  \frac{1}{X}\frac{d}{dt} X(t) = - \beta \ln X(t).
\end{equation}
In this limit, the relative growth goes to infinity as $\lim_{t\to 0}$, which could be seen as a verification of the maximally coherent characteristic of the system. One is reminded of a system that is perturbed and left to reach a new equilibrium. Petroni's definition of maximal coherence in the context of the Gompertz equation was shown to rigorously correspond to the definition of coherence used in quantum mechanical systems \cite{molski2003coherent} attributed to Glauber (1963) \cite{glauber1963coherent}.

Another way to generalize the logistic equation is to augment the resource terms to higher orders as follows,
\begin{equation}
  \frac{1}{X}\frac{d}{dt} X(t) = \beta \left[(1 - X(t)) + \frac{1}{2}(1 - X(t))^2 + ... \frac{1}{K}(1 - X(t))^K\right],
\end{equation}
where fractional prefactors to each term will be justified in the sequel.
After some algebra (Appendix \ref{appendix:a}) and letting $\beta=1$ for simplicity, this augmented logistic equation reduces to a closed form expression involving Gauss' Hypergeometric function, ${}_1F_{2}$, viz.
\begin{equation}
 \label{eq:modLogistic}
\frac{d}{dt}\ln{X(t)} = - \frac{{}_{1}F_{2}({1,K+1,K+2;1-X(t)})}{K+1}(1-X(t))^{K+1} - \ln{X(t)}.
 \end{equation}

 In the context of higher order terms, the equivalent of $\theta\rightarrow 0$ in $\theta$-logistic growth is that the higher order terms go to infinity, i.e. $K \rightarrow \infty$, which again reduces to the Gompertz equation by virtue of the Hypergeometric term going to zero. 

 The emergence of Gompertz growth from higher order logistic growth can be seen as another way to encode non-linear effects that were seen as coherence in accordance with Petroni and Molski. In practice, the augmented logistic model has a faster growth rate decay than the familiar $\theta$-logistic growth (Figure \ref{fig:hypergeometric_plain}). If we interpret deceleration of relative growth as degree of coherence, the augmented logistic growth would be more coherent than the $\theta$-logistic model for a given starting value. 

\insertPdfFig{hypergeometric_plain}{Comparison relative growth rate in the Richards $\theta$-logistic model (Eq. \ref{eq:Rich}), Gompertz model (Eq. \ref{eq:Gomp}), and the augmented logistic model (Eq. \ref{eq:modLogistic}), all as a function of the abundance variable. Both dependent and independent variables are normalized so that final abundance (or size) is set to unity. The Richards $\theta$-parameter is annotated in brackets in the legend, in addition to the number of terms used in the augmented logistic model.} 

In spite of the connection of the Gompertz model with higher order terms in an augmented logistic model, it is not clear whether the source of growth is external or internal to the system. The right hand side ``resource'' term is not specific enough so as to specify the location of the source. One could argue that the since the Gompertz model has asymptotically infinite initial growth rate, the system could be seen as exposed to a ubiquitous signal (or catalyst in Markov's language) that initiates the growth and subsequently decays exponentially like a memoryless process. To further explore the source of growth we turn to the microscopic domain.

\section{Microscopic Growth}
\label{sec:micro}

\subsection{Microscopic Logistic Growth}
Consider $N$ entities each represented by a set of independent and identically distributed random abundance variables, $x_i(t), \quad i=1, ..., N$, which are unitless and take values between 0 and 1, where 0 represents no realized growth while 1 represents the maximum potential growth having been realized and $\mu$ representing the mean of the distribution. With these variables it will also be convenient to also define an \textit{abundance potential} variable, $s_i = 1 - x_i$, where we omit the temporal variable for ease of exposition.

Now, consider a network between these $N$ entities along a graph with a set of edges and nodes which delineates the interaction of the growth variables, where we define interaction in this context as the transfer of one entity's potential growth to another entity which converts potential to realized growth. Then consider the growth equation with only pairwise interactions, or \textit{first order} interactions, governed by an adjacency matrix ${a_{ij}}$ as a binary matrix with ones where the $i^{th}$ and $j^{th}$ nodes are connected, and zeroes otherwise, viz.
\begin{equation}
\label{eq:modSIR}
\frac{d x_i}{dt} = x_i\sum_{j \in \mathcal{N}(i)}{a_{ij}}s_{j} \quad \forall i,
\end{equation}
where $\mathcal{N}(i)$ is the neighborhood of nodes connected to node $i$.
First note that for a single pair this would amount to traditional logistic growth. We choose to work in a unitless domain without parameters for the interaction strength and carrying capacity for simplicity of exposition. These can be added without changing the conclusions of these arguments. 

Now as we let N grow, we impose a simple rule: \textit{Each entity can only receive growth potential from a single entity and create growth potential in a single possibly different entity}. Using this rule and rearranging and summing across all entities,
\begin{equation}
\label{eq:modSIR2}
\frac{d}{dt} \ln \left[ \prod_i^N x_i \right ] = \sum_j^N s_{j}.
\end{equation}

Dividing by the total number of entities and with our assumption of the variables being independent and identically distributed, we obtain the simplest form of logistic growth of the geometric mean of the ensemble driven by the sample mean of the inverse abundance, or the \textit{potential abundance}, 
\begin{equation}
\label{eq:modSIR3}
\frac{d}{dt} \ln \left[ \prod_i^N x_i \right ]^{1/N} = \frac{1}{N}\sum_j^N s_j 
\end{equation}

As N grows, both of the geometric and the arithmetic means converge in expectation to their respective distributional parameters, and thus we can identify these macroscopic quantites starting from pairwise microscopic principles. Specifically, letting $\tilde{x}(N)$ represent the sample geometric mean of $x_i$ and $\bar{x}(N)$ represent the sample arithmetic mean, each converging to their distributional counterparts $\gamma$ and $\mu$, we have 
\begin{eqnarray*}
  \lim_{N\rightarrow \infty} \frac{d}{dt} \ln \left[ \prod_i^N x_i \right ]^{1/N} &=& \lim_{N\rightarrow \infty} \frac{1}{N}\sum_i^N s_i \\\
  \lim_{N\rightarrow \infty} \frac{d}{dt} \ln \tilde{x}(N) &=& \lim_{N\rightarrow \infty}(1 - \bar{x}(N)) \\
  \frac{d}{dt} \ln \gamma &=&(1 - \mu),
  % \frac{d}{dt} \ln m &=&- \ln \bar{s},
\end{eqnarray*}
where the last equality follows from a convergence in expectation.

We make special note that logistic growth emerges from pairwise interactions only. This does not mean that the logistic growth cannot emerge from other microscopic settings but it nevertheless provides one possible scenario under which the logistic curve can emerge. Now, as we shall allow higher order interactions we will see the Gompertz curve emerge.

\subsection{Microscopic Gompertz Growth}
\label{sec:microGomp}

Instead of limiting the growth equation to pairwise interactions, we allow higher order interactions in the form of triplets, quadruplets, and so on. Analogous to the pairwise exclusivity rule, each entity can again only participate in \textit{one} interaction group at any given order and any given direction. For example, any triplet can only interact with one other triplet and have exclusive possession of its entities at the triplet level (Figure \ref{fig:order}).

\insertPngFig{order}{Exclusive interaction groups at each order: A) First order interaction, each entity is paired with a single other entity. B) Second order interaction, mutually exclusive groups of two entities interact with each other. B) Third order interaction, same as B) but for groups of three entities.} 

Thus, let $S_k$ be the set of all mutually exclusive $k$-sized groups where $k=1,..., N/2$, where the highest order are capped at $N/2$ since that would be the largest number of entities in a single group that could interact with an identically sized group. This means that we consider the geometric mean of each $k$-group as follows,
\begin{equation}
\label{eq:modSIR4}
\frac{d}{dt} \ln x_{i_1} x_{i_2} \dots x_{i_k} = s_{j_1}s_{j_2}\dots s_{j_k}, 
\end{equation}
where the indices $i$ and $j$ represent the higher order interaction indices between the mutually exclusive groups of entities for each $k$ from 1 and up to $N/2$. Thus, the aggregated growth is represented by the sum of each interaction order,
\begin{equation}
\label{eq:MicroGomp0}
\frac{d}{dt} \ln \left[ \prod_i^N x_i \right ] = \sum_{j\in S_1}^N s_{j} + \sum_{j_1, j_2 \in S_2} s_{j_1}s_{j_2} + ... + \sum_{j_1, j_2, ..., j_{N/2} \in S_{N/2}} s_{j_1}s_{j_2}\dots s_{j_{N/2}},
\end{equation}
where $S_k$ is the set of all mutually exclusive $k-$sized groups. In each of the sums the total number of terms is equal to the total number of groups which at each order is $N/k$ where $k$ is the interaction order. Thus, if we divide the right hand side by $N$, we can identify a convergence property as both $k$ and $N$ grows:
\begin{enumerate}
\item As the number of entities grow each term will converge in expectation to higher powers of the mean of $s_i$, multiplied by the prefactor $1$, $1/2$, $1/3$, etc., 
\begin{equation}
\label{eq:MicroGomp11}
\lim_{N\rightarrow \infty} \frac{1}{N}\sum_{j_1, j_2, ..., j_k \in S_k} s_{j_1}s_{j_2}\dots s_{j_k} = \frac{1}{N}\frac{N}{k} \mu^{k} = \frac{1}{k} \mu^{k}.
\end{equation}
\item As the higher order interaction terms grow
\begin{equation}
\label{eq:MicroGomp12}
\lim_{k\rightarrow \infty} \frac{1}{N}\sum_{j_1, j_2, ..., j_k \in S_k} s_{j_1}s_{j_2}\dots s_{j_k} = 0
\end{equation}
\end{enumerate}
Thus, as each higher order term is composed of a decreasing number of summations and an increasing number of multiplications, the product of $s_i$ goes to zero since each random variable is $iid$ and bounded between 0 and 1. 
Now, using the Taylor series identity $\mu+\mu^2/2+... = -\ln(1-\mu)$, we obtain Gompertz growth in terms of the arithmetic and the geometric means of the microscopic entities
\begin{eqnarray}
\label{eq:MicroGomp13}
\lim_{N\rightarrow \infty}\frac{d}{dt} \ln \left[ \prod_i^N x_i \right ]^{1/N} &=&  \lim_{N\rightarrow \infty}\left[\sum_j^N s_{j} + \sum_{j_1, j_2 \in S_2} s_{j_1}s_{j_2} + ...\right]\\
\frac{d}{dt} \ln \gamma &=&  \mu + \frac{1}{2} \mu^2 + ...\\
\frac{d}{dt} \ln \gamma &=& - \ln (1-\mu).
\end{eqnarray}

Note that for the right hand side to exhibit asymptotically infinite growth rate at the onset of growth, it is not only sufficient but also necessary to have higher order terms. This is seen by virtue of the infinite polynomial representation of the logarithm and implies that Gompertz growth cannot result from pairwise interactions under the assumptions presented in this model, but rather requires all higher order interactions to be present.

Since all higher order interaction would necessarily imply non-local interaction, the natural interpretation would be that the growth is activated by a ubiquitous field accessible by all entities simultaneously. 
In this way one could think of the activating field as an \textit{effectively} external field, even though it is possible that the field emerges as a collective emergent phenomenon purely from within the system or organism.

Another way to see the source of growth as effectively external is to recognize that the source of growth in the higher order terms is coming from the abundance potential and not the realized potential, while in the logistic growth model this is not necessarily the case. 
To see this, first swap the role of $x_i$ and $s_i$ on the right hand side of Equation \ref{eq:modSIR} to see that the same result can be obtained under the exclusivity condition. 
Then note that when augmenting the growth equation with higher order terms on the right-hand side of the same equation, the same swapping will not lead to the Gompertz model. 

This causality reversal is indeed what is done when modelling communicable diseases in a microscopic network context, where $x_i$ represents the infected nodes and $s_i$ represents the susceptible nodes (see e.g. \citet{estrada2022networked}). But because higher order growth does not allow this causality reversal, where the infected nodes are the source of spread, the Gompertz model cannot be used since it would imply that the susceptible nodes are the source of the disease spreading, which would turn communicable disease theory on its head. In this context, the Gompertz model would rather suggest a system disturbed by a ubiquitous, simultaneous, and non-local stressor eliciting a corresponding stress response through which a new stress tolerance baseline is gradually established.

\section{Unifying the Gaussian and Exponential curves with the Logistic and the Gompertz curves}

Up until now we have seen how logistic growth emerges from pairwise correlated growth while Gompertz growth emerges from infinitely correlated growth. These are both growth modes where there is some degree of connectivity between the entities. But what if there is no connectivity between the entities but each entity is independent? In such a case we can formulate a growth equation where the relative growth is only linearly dependent on time, where we make a slight modification and model the \textit{potential abundance}, $s(t)$, instead of the realized abundance,
\begin{equation}
  \frac{d}{dt}s(t) = - \beta t s(t),
\end{equation}
or 
\begin{equation}
  \frac{d}{dt}\ln s(t) = - \beta t,
\end{equation}
for some constant $\beta>0$. 
 
Solving this equation yields the familiar Gaussian function, 
\begin{equation}
  s(t) = A e^{ - \beta t^2 / 2},
\end{equation}
where $\beta$ can be interpreted as the inverse variance or dispersion of the growth phenomena and where $A=1$ when $s(0)=1$. Thus, not surprisingly Gaussian growth emerges for mutually independent events that have a linear time dependence. The same result was originally formulated by Gauss with space as the independent variable instead of time and with error as the dependent variable instead of abundance, all in the context of the prediction of the heavenly object called Ceres \cite{gauss1857theory} (see \cite{stahl2006evolution} for details). In the context of measurement error, one would equate the above equation to the statement that the relative measurement error decreases linearly with size. 

If we remove the dependency on time altogether, we observe the exponential growth equation
\begin{equation}
  \frac{d}{dt}\ln s(t) = - \beta,
\end{equation}
which yields 
\begin{equation}
  s(t) = A e^{ - \beta t}.
\end{equation}
This exponential equation implies a complete independence on both time and of the mutual entities involved in the growth. In the context of measurement error this function would be the Laplacian error which states that absolute measurement error decreases linearly with the error itself \footnote{Or as Laplace originally stated: ``[...] we have no reason to suppose a different law for the ordinates than for their differences'' \cite{stahl2006evolution}.}. 
% Note also that for both the Gaussian and the exponential growth functions, the \textit{realized abundance}, $x(t)$, could be substituted as the dependent variable where the physical interpretation would be a decay process. 

In general, our choice of time as the dependent variable can be substituted for any suitable metric of choice. Another example of space as the dependent variable in the Gompertz setting is given by \citet{molski2003coherent}.

\subsection{Unification with the Poisson distribution}

As a side note, the Poisson distribution can be included in this discussion as the distribution which governs the infinitesimal growth rate of a coherent or infinitely correlated group of entities. By letting the growth rate be the rate at which a zero-count is observed with the Poisson mean parameter equal to $-\beta t$, one recovers the Gompertz model in Equation \ref{eq:Gomp}. This argument was put forth by \citet{shklovskii2005simple}, and \citet{glauber1963coherent} was the first to point out the connection where the Poisson distribution reflects a coherent system. Glauber also showed that incoherent systems, or systems with independent entities, follow the Gaussian distribution which is in accordance with our results and discussion here. The mapping between coherence and incoherence has also been postulated as a medical diagnostic technique by \citet{zhang1994log}, where incoherence is a sign of disease while coherence is a sign of well-being.

\section{Discussion and Conclusion}
\label{sec:discussion}

To summarize, we have seen that both time independent and uncorrelated growth yields the exponential growth function, time dependent and uncorrelated growth yields Gaussian growth, time independent and pairwise correlated growth yields the logistic function and time independent and infinitely correlated growth yields the Gompertz function. 

We have further shown how Gompertz growth emerges both at the microscopic and the macroscopic level as a non-local effect. A novel bridge between logistic and Gompertz growth is established involving Gauss' Hypergeometric function that serves as a replacement for the Richards model and is derived from first principles. This bridge presupposes the source of growth to be the potential abundance as opposed to the realized abundance. On the microscopic level, the Gompertz model emerges through an exclusivity rule at every order of interaction. As a corollary, this thesis shows how the logistic function is the result of local growth processes with pairwise-interactions only, in line with its observation in the fields of predator-prey models and in communicable diseases \cite{may2007theoretical}. 

An infinitely interacting system where entities are classical in nature with well-defined position and momentum, breaks the law of causality unless we allow the source of growth to be a ubiquitous field, externally or internally created, which affects all microscopic entities instantaneously \footnote{Or at speeds much greater than the differential times in the growth equations}. With a such a source, entities need not be interacting with each other but rather respond to the field-induced stimuli. Examples of such models are already given by De Lauro (2014), Sacher (1956) and Markov (2019). \citet{de2014stochastic} and \citet{sacher1956statistical} consider the source of growth to be stochastic in nature, and while Lauro et al. do not discuss the external nature of such a source, their model does not preclude it. More generally, with growth as stochastically rooted it is also ubiquitously present uniformly across all entities from the very onset of growth. Markov (2019)\cite{markov2019reaction} discusses growth within the context of chemical reactions, where the source of growth is a catalytic agent that is implicitly available ubiquitously to all reactants to the same degree, exactly in line with De Lauro and Sacher's models. 

For many growth processes within biological organisms, it has been well established that the mediator is the electromagnetic field \cite{becker1984electromagnetic}, which suggests that the accompanying growth pattern in these processes follows the Gompertz model as the result of a ubiquitous growth-inducing field. For growth in larger ecological or societal systems other catalysts can be air pollution, oxygen deprivation, or even instant psychologically induced catalysts. 

We finally note that the original formulation of the Gompertz model made by Gompertz himself was concerning mortality rates as a function of age. He noticed that mortality rates increased at a geometric rate with age. In this context the ubiquitous stressor postulated above would be the living conditions on our planet, for which time acts as a proxy. One could further postulate this stressor as a field surrounding the earth, e.g. in its electromagnetic footprint often related with the Schumann Resonance \cite{cherry2002schumann,schumann1952strahlungslosen}. It is indeed curious that we do not see mortality rates behave as linearly dependent on time, for which a Gaussian curve would be appropriate. Nor do we see any finite-ordered polynomial time dependence on mortality rates. Instead we see growth of mortality rates as sourced from other entities at all interaction orders, which in light of our discussion above is equivalent to mortality coming from an environmental stressor.

\section*{Competing interests}
The authors declare no competing interests.

% \section*{Author Contributions}
% M. A. Haugen: Conceptualization, Methodology, Software, Writing. D. Gilbert: Conceptualization, Reviewing and Editing.

\section{Appendix A}
\label{appendix:a}
Here we arrive at a closed-form expression for a finite number of interaction terms in this modified logistic model, which represents the infection term in the SIR model. Working with a single unitless growth variable and omitting the multivariate network without loss of generality, we start with the logistic growth equation augmented by $K$ terms and using the nomenclature defined in Section \ref{sec:micro}, viz. 

\begin{equation}
\frac{d \ln{x}}{dt} = \sum_{i=1}^K s^i/i.
\end{equation}
Using standard integrals we see that this sum amounts to a Gompertz growth term adjusted by a Hypergeometric function,
\begin{eqnarray*}
  \sum_{i=1}^K s^i/i &=& \int_s \sum_{i=0}^{K-1} s^i ds\\
  &=&\int_s \frac{1-s^K}{1-s} ds\\
  &=&- \frac{{}_{1}F_{2}({1,K+1,K+2;s})}{K+1}s^{K+1} - \ln(1-s),
\end{eqnarray*}
where ${}_1F_{2}$ is Gauss' Hypergeometric function, and integration limits are indefinite. In summary, the modified logistic function emerges as
 \begin{equation}
 \label{eq:modLogistic2}
\frac{d}{dt}\ln{x(t)} = - \frac{{}_{1}F_{2}({1,K+1,K+2;1-x(t)})}{K+1}(1-x(t))^{K+1} - \ln{x(t)},
 \end{equation}
 which represents the novel bridge between logistic and Gompertz growth. Note that for $K=1$, this relative growth rate becomes the standard logistic function, while increasing the number of higher order terms will move the equation closer to the Gompertz curve as the Hypergeometric term approaches zero. This modification to the logistic function does not have a 1-to-1 correspondence with the Richards model, but serves instead as an alternative when encoding non-linear or collaborative growth effects (see Fig. \ref{fig:hypergeometric_plain}). 


% Estimating the parameters using real data requires some consideration around the variance of the growth variable as a function of time. This variance will in turn inform the estimation process by considering a weighted minimization of the residuals of the observations regressed onto the model. In the stochastic Gompertz process postulated in Section \ref{seq:alt}, the process under a log-transformation is the Ornstein-Uhlenbeck process which has a closed from expression for its transition probability, and thus its variance as a function of time. This variance can then be used in a weighted least squares estimation technique or any other weighted technique. In the original domain there is also a closed form expression available but it is somewhat more convoluted and given that observations in this domain are visually quite homoscedastic with equal variance across time (Fig \ref{fig:Gompertz_vs_SIRD} and Tables \ref{tab:gomp} and \ref{tab:sird}), this is expression and consideration is omitted. But for the log-domain, observations exibit changing variance with time. 

% To be explicit, if we consider the notation from Section \ref{seq:alt} and let $P(z, t|z', t')$ be the transition probability from some growth and time coordinate $z', t'$ to some other growth and time coordinate $z, t$, than the transition probability of the Olstein-Uhlenbeck process in \ref{eq:microscopic} is given by\cite{risken1996fokker},
% \begin{equation}
% P(z, t|z', t') = \sqrt{\frac{\beta}{2\pi \sigma (1 - e^{-2\beta (t-t')})}}\exp{\left[ -\frac{\beta (z - e^{-\beta(t-t')}z')^2}{2\sigma(1-e^{-2 \beta(t-t')})} \right]}.
% \end{equation}

% If we consider $t$ the time at which the stationary distribution is reached and that this stationary distribution is given as a delta function, variance of the growth will be given as
% \begin{equation}
% \text{Var[Z(t)|t']} = \sigma(1-e^{-2 \beta(t-t')}).
% \end{equation}

% Because weights on observations in a regression are invariant to multiplication we can let $\sigma=1$. The slope coefficient would have to be estimated either by a standard least squares or through the 3-parameter model in (\ref{eq:GompertzODE}).


% \insertPdfFig{Gompertz_vs_SIRD_gompfit_log_domain}{Log transformed cumulative diseased plotted against its time derivative after $Y(t) / Y_{max}>0.005$, with a straight line fit regressing onto the $50^{th}$ quantile.} 

%  Carletti (left) transformation to our notation (right)
% gamma - eta
% eta - xi
% beta - kappa 


\bibliographystyle{abbrvnat}
\bibliography{references2}  %%% Uncomment this line and comment out the ``thebibliography'' section below to use the external .bib file (using bibtex) .

%%% Uncomment this section and comment out the \bibliography{references} line above to use inline references.
% \begin{thebibliography}{1}

%   \bibitem{kour2014real}
%   George Nour and Raid Saabne.
%   \newblock Real-time segmentation of on-line handwritten arabic script.
%   \newblock In {\em Frontiers in Handwriting Recognition (ICFHR), 2014 14th
%       International Conference on}, pages 417--422. IEEE, 2014.

%   \bibitem{kour2014fast}
%   George Nour and Raid Saabne.
%   \newblock Fast classification of handwritten on-line arabic characters.
%   \newblock In {\em Soft Computing and Pattern Recognition (SoCPaR), 2014 6th
%       International Conference of}, pages 312--318. IEEE, 2014.

%   \bibitem{hadash2018estimate}
%   Guy Hadash, Einat Nermany, Boaz Carmeli, Ofer Lavi, George Nour, and Alon
%   Jacovi.
%   \newblock Estimate and replace: A novel approach to integrating deep neural
%   networks with existing applications.
%   \newblock {\em arXiv preprint arXiv:1804.09028}, 2018.

% \end{thebibliography}


\end{document}