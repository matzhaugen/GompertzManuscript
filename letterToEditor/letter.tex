\documentclass[11pt,a4paper,roman]{moderncv}      
\usepackage[english]{babel}

\moderncvstyle{classic}                            
\moderncvcolor{black}                            

% character encoding
\usepackage[utf8]{inputenc}                     

% adjust the page margins
\usepackage[scale=0.73]{geometry}

% personal data
\name{Matz Andreas Haugen}{}
\email{matzhaugen@gmail.com}
\address{Enerhauggata 1, 0651, Oslo, Norway}


\begin{document}

\recipient{To}{Editor, \\
               Proceedings of the National Academy of Sciences,\\ 
               }
\date{\today}
\opening{\textbf{Initial submission of manuscript titled: ``Unifying Gompertzian growth with the communicable disease spreading paradigm''}}
\closing{Sincerely yours, \vspace{-1em}}


\enclosure[Enclosed ]{
              \\ List of referees
%                  \hspace{0.0em} 5. GATE (2016) \hspace{2em} 9. Identity-Proof (Voter-Id)\\
%                  2. Diploma (MarkSheet/Certificate)
%                  \hspace{1.8em} 6. SLET (2021) \hspace{1.9em} 10. Resume\\        
%                  3. B.Tech (MarkSheet/Certificate)
%                  \hspace{2.3em} 7. UGC-NET (2022) \hspace{-0.1em} \\
%                  4. M.Tech (MarkSheet/Certificate)
%                  \hspace{2.1em} 8. Caste Certificate \\
                 }
\makelettertitle



Respected Editor,
\\
%references such as what and how you got this information
\vspace{1em}
We hereby submit a manuscript where we seek to connect the communicable disease paradigm with the recently observed mortality patterns of the initial Coronavirus pandemic in March-April 2020 that have been shown by many to exhibit Gompertzian growth. 
In the manuscript, we show that this type of growth pattern is incompatible with traditional communicable disease spreading models, i.e. the SIR (Susceptible-Infected-Recovered) model family of Kermack and McKendrick. Instead, the observed patterns can cleanly be explained by a simpler model without the need for a disease propagating stage, but rather through a ubiquitous stressor which elicits an instantaneous and mutual stress response, amounting to a 2-parameter model. 
The mathematical thesis is based on a simple Stochastic Differential Equation where the stress response is a random process, interpretable both at the macroscopic and the microscopic level. 
We also show a remarkable connection between coherent behavior previously relegated to microscopic quantum domains, now exhibited in national mortality patterns. In light of this, we equate one of the model parameters in the traditional disease models with the level of \emph{coherency} of growth, where coherency has been rigorously defined in the physics literature (see manuscript).

The findings of this paper call for a fundamental and interdisciplinary discussion of our accepted knowledge on communicable diseases, as the observations constitute a classic Kuhnian "anomaly" suggesting a paradigm shift away from that of a purely communicable paradigm to a hybrid where the environment plays a bigger role.

If you do decide to review this paper you may find the enclosed list of possible referees helpful. 
Some of them are listed in the references of the manuscript. 

Further, as we wish to have a double-blind review process, our names are omitted in the manuscript.

Thank you for your consideration.

\makeletterclosing
\newpage
\textbf{List of possible referees}
\begin{enumerate}
\item N. El Karoui, mathematics, stochastic models
\item M. Molski, physics, Gompertz and quantum systems
\item W. Whitt, mathematics, stochastic models
\item N. C. Petroni, E. De Lauro, physics, Gompertz
\item E. B. Postnikov, physics, SIR models
\item G. Vattay, physics, SIR models
\item M. Levitt, computational biology, Gompertz models
\item E. Estrada, mathematics, graph theory, SIR models
\item T. Carletti, D. Fanelli, and F. Piazza, physics, SIR models
\item Z. Bajzer, biology, has a very good review paper on Gompertz modeling 
\item J. C. Mombach, physics, growth models
\item B. Shklovskii, physics, theory of Gompertz
\item D. Smilkov, C. or L. Kocarev, physics and mathematics, SIR models
\item C. G. Antonopoulos, physics, chaos and SIR models
\item K. Rypdal and M. Rypdal, mathematics, SIR models

\end{enumerate} 

\end{document}
