\documentclass[11pt,a4paper,roman]{moderncv}      
\usepackage[english]{babel}

\moderncvstyle{classic}                            
\moderncvcolor{black}                            

% character encoding
\usepackage[utf8]{inputenc}                     

% adjust the page margins
\usepackage[scale=0.73]{geometry}

% personal data
\name{Matz Andreas Haugen}{}
\email{matzhaugen@gmail.com}
\address{Valhallveien 13, 0196, Oslo, Norway}


\begin{document}

\recipient{To}{Editor, \\
               Journal of Physics A: Mathematical and Theoretical,\\ 
               }
\date{\today}
\opening{\textbf{Initial submission of manuscript titled: ``Unifying the communicable disease spreading paradigm with Gompertzian growth''}}
\closing{Sincerely yours, \vspace{-1em}}


\enclosure[Enclosed ]{
              \\ List of two additional referees 
%                  \hspace{0.0em} 5. GATE (2016) \hspace{2em} 9. Identity-Proof (Voter-Id)\\
%                  2. Diploma (MarkSheet/Certificate)
%                  \hspace{1.8em} 6. SLET (2021) \hspace{1.9em} 10. Resume\\        
%                  3. B.Tech (MarkSheet/Certificate)
%                  \hspace{2.3em} 7. UGC-NET (2022) \hspace{-0.1em} \\
%                  4. M.Tech (MarkSheet/Certificate)
%                  \hspace{2.1em} 8. Caste Certificate \\
                 }
\makelettertitle



Respected Editor,
\\
%references such as what and how you got this information
\vspace{1em}
We hereby submit a manuscript where we seek to connect the communicable disease paradigm with the recently observed mortality patterns of the initial Coronavirus pandemic in March-April 2020 that have been shown by many to exhibit Gompertzian growth. 
In the manuscript, we show that this type of growth pattern is incompatible with traditional communicable disease spreading models, i.e. the SIR (Susceptible-Infected-Recovered) model family of Kermack and McKendrick. Instead, the observed patterns can be explained by a simpler model without the need for a disease propagating stage, but rather through a ubiquitous stressor which elicits an instantaneous and mutual stress response, amounting to a 2-parameter model. 
The mathematical thesis is based on a simple Stochastic Differential Equation where the stress response is a random process, interpretable both at the macroscopic and the microscopic level. 
We also show a remarkable connection between coherent behavior previously relegated to microscopic quantum domains, now exhibited in national mortality patterns. In light of this, we equate one of the model parameters in the traditional disease models with the level of \emph{coherency} of growth, where coherency has been rigorously defined in the physics literature (see manuscript).

The findings of this paper call for a fundamental and interdisciplinary discussion of our accepted knowledge on communicable diseases, as the observations constitute a classic Kuhnian "anomaly" suggesting a paradigm shift away from that of a purely communicable paradigm to a hybrid where the environment plays a bigger role.

Due to the conclusions of our paper, I request that this paper receives more reviews than your average, and at least 3 reviews. Even though we have recommended a few reviewers, perhaps you know someone who is better qualified to review, especially since we are submitting to a special section of your journal.

Thank you for your consideration.

\makeletterclosing
\newpage

\begin{enumerate}
\item M Kröger, Polymer Physics, Department of Materials, ETH Zurich, Zurich CH-8093, Switzerland, mk@mat.ethz.ch
\item R Schlickeiser, Institut für Theoretische Physik, Lehrstuhl IV: Weltraum- und Astrophysik, Ruhr-Universität Bochum, D-44780 Bochum, Germany, rsch@tp4.rub.de
\end{enumerate}

% \textbf{List of Editorial board members}
% \begin{enumerate}
% \item Levine, Herbert
% \item Levin, Simon A.
% \item Marquet, Pablo A.
% \end{enumerate}

% \textbf{List of NAS members (in order of relevance)} 
% \begin{enumerate}
% \item Levine, Herbert
% \item Levin, Simon A.
% \item Michael Levitt 

% \item Nigel D. Goldenfeld 
% \item Evans, Steven N.
% \item Altshuler, Boris L.
% \end{enumerate}
% \textbf{List of possible referees}
% \begin{enumerate}
% % IOP REVIEWERS
% \item Martin Evans,Professor of Statistical Physics,  https://www2.ph.ed.ac.uk/~mevans/, m.evans@ed.ac.uk
% \item Prof Ralf Metzler, Institute of Physics & Astronomy
% University of Potsdam
% Karl-Liebknecht-Str 24/25, Haus 28
% D-14476 Potsdam-Golm 
% \item Michael Victor Berry, U of Bristol, Prof. of Physics, asymptotico(add at bristol.ac.uk) 
% \item Ronald Dickman, Professor of Physics, UFMG, dickman@fisica.ufmg.br
% \item Ewa Gudowska-Nowak, ewa.gudowska-nowak[at]uj.edu.pl , 
% Marian Smoluchowski
% Institute of Physics,
% Jagiellonian University
% \item Shamik Gupta Tata Institute of Fundamental Research, Mumbai, India, shamik.gupta-At-theory.tifr.res.in (Official)
% \item Alain Barrat, CNRS & Aix-Marseille Université, France , Alain.Barrat@cpt.univ-mrs.fr
% \item Daniel ben-Avraham, dbenavra@clarkson.edu, Professor of Physics, Clarkson University
% \item Tom Britton, Dean of Mathematics and Physics, Stockholm University, tom.britton@math.su.se
% \item Mark Lewis, marklewis@uvic.ca, Department of Mathematics and Statistics, University of Victoria
% \item Alexandru Hening, Department of Mathematics at Texas A and M University,  ahening@tamu.edu
% \item K-Y. Lam, Mathematics,  the Ohio State University, Columbus, lam.184@osu.edu 
% \item X. Zou, Western University, London, Ontario, xzou@uwo.ca
% \item S. Petrovskii, Mathematical Sciences, University of Leicester, sp237@le.ac.uk
% \item Pierre Magal, Institut de Mathématiques de Bordeaux, Université de Bordeaux, pierre.magal@u-bordeaux.fr 
% \item Vladimir N. Binhi, vnbin@mail.ru, Prokhorov General Physics Institute of the Russian Academy of Sciences, 38 Vavilov St., 119991 Moscow, Russia
% Is familiar with the biological effects due geomagnetic disturbances and fields.
% \item Andrei B. Rubin, rubin@biophysics.msu.ru, Faculty of Biology, Lomonosov Moscow State University, Leninskie Gory 1/12, 119234 Moscow, Russia
% Is familiar with the biological effects due geomagnetic disturbances and fields.
% \item Viacheslav V. Krylov, I.D. Papanin Institute for Biology of Inland Waters Russian Academy of Sciences, Borok, Russian Federation, krylovviacheslav@mail.ru
% Is intimately familiar with geomagnetic disturbances' effect on biology.
% \item Marcin Molski, mamolski@amu.edu.pl, Adam Mickiewicz University, physics, Gompertz and quantum systems. Has done work on connecting the gompertz model to quantum coherent systems.
% \item Francesco Piazza, Francesco.Piazza@cnrs-orleans.fr, Max Planck Institute for the Physics of Complex Systems, physics, SIR models.
% Has written one of the central papers modeling covid disease propagation from the SIR perspective. We have based our methods on their paper.
% \item Robert M May, robert.may@zoo.ox.ac.uk, Professor of Zoology, University of Oxford.
% \end{enumerate}


% Other possible reviewers:
% \begin{enumerate}
% \item Joel Miller
% \item Mark Chaplain
% \item Vishwesha Guttal
% \item Natalia Komorova
% \item Shigui Ruan
% \item Simon Levin

% \item Helen Wearing
% \item James Watmough
% \item Jane Heffernan
% \item Jing-an Cui
% \item Julien Arino

% \end{enumerate} 

\end{document}
