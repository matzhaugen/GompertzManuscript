\documentclass[11pt,a4paper,roman]{moderncv}      
\usepackage[english]{babel}

\moderncvstyle{classic}                            
\moderncvcolor{black}                            

% character encoding
\usepackage[utf8]{inputenc}                     

% adjust the page margins
\usepackage[scale=0.73]{geometry}

% personal data
\name{Matz Andreas Haugen}{}
\email{matzhaugen@gmail.com}
\address{Enerhauggata 1, 0651, Oslo, Norway}


\begin{document}

\recipient{To}{Editor, \\
               Chaos, Solitons, and Fractals,\\ 
               }
\date{\today}
\opening{\textbf{Initial submission of manuscript titled: ``Unifying the communicable disease spreading paradigm with Gompertzian growth''}}
\closing{Sincerely yours, \vspace{-1em}}


\enclosure[Enclosed ]{
              \\ List of referees
%                  \hspace{0.0em} 5. GATE (2016) \hspace{2em} 9. Identity-Proof (Voter-Id)\\
%                  2. Diploma (MarkSheet/Certificate)
%                  \hspace{1.8em} 6. SLET (2021) \hspace{1.9em} 10. Resume\\        
%                  3. B.Tech (MarkSheet/Certificate)
%                  \hspace{2.3em} 7. UGC-NET (2022) \hspace{-0.1em} \\
%                  4. M.Tech (MarkSheet/Certificate)
%                  \hspace{2.1em} 8. Caste Certificate \\
                 }
\makelettertitle



Respected Editor,
\\
%references such as what and how you got this information
\vspace{1em}
We hereby submit a manuscript where we seek to connect the communicable disease paradigm with the recently observed mortality patterns of the initial Coronavirus pandemic in March-April 2020 that have been shown by many to exhibit Gompertzian growth. 
In the manuscript, we show that this type of growth pattern is incompatible with traditional communicable disease spreading models, i.e. the SIR (Susceptible-Infected-Recovered) model family of Kermack and McKendrick. Instead, the observed patterns can be explained by a simpler model without the need for a disease propagating stage, but rather through a ubiquitous stressor which elicits an instantaneous and mutual stress response, amounting to a 2-parameter model. 
The mathematical thesis is based on a simple Stochastic Differential Equation where the stress response is a random process, interpretable both at the macroscopic and the microscopic level. 
We also show a remarkable connection between coherent behavior previously relegated to microscopic quantum domains, now exhibited in national mortality patterns. In light of this, we equate one of the model parameters in the traditional disease models with the level of \emph{coherency} of growth, where coherency has been rigorously defined in the physics literature (see manuscript).

The findings of this paper call for a fundamental and interdisciplinary discussion of our accepted knowledge on communicable diseases, as the observations constitute a classic Kuhnian "anomaly" suggesting a paradigm shift away from that of a purely communicable paradigm to a hybrid where the environment plays a bigger role.

If you do decide to review this paper you may find the enclosed list of possible referees helpful. 
Some of them are listed in the references of the manuscript. 

Thank you for your consideration.

\makeletterclosing
\newpage

% \textbf{List of Editorial board members}
% \begin{enumerate}
% \item Levine, Herbert
% \item Levin, Simon A.
% \item Marquet, Pablo A.
% \end{enumerate}

% \textbf{List of NAS members (in order of relevance)} 
% \begin{enumerate}
% \item Levine, Herbert
% \item Levin, Simon A.
% \item Michael Levitt 

% \item Nigel D. Goldenfeld 
% \item Evans, Steven N.
% \item Altshuler, Boris L.
% \end{enumerate}
\textbf{List of possible referees}
\begin{enumerate}

\item Nicola Cufaro Petroni, cufaro@ba.infn.it,
Dipartimento di Matematica and TIRES, University of Bari (Ret).
Has done extensive work on the Gompertz model in relation to the Richard's model, connecting it with stochastic growth models and the Logistic (Lotka-Volterra) model. We use their concept of growth rooted in stochastic phenomena.
\item Salvatore De Martino,
demartino247@gmail.com, 
Dipartimento di Ingegneria dell’Informazione ed Elettrica.
Same as above
\item Marcin Molski, mamolski@amu.edu.pl, Adam Mickiewicz University, physics, Gompertz and quantum systems. Has done work on connecting the gompertz model to quantum coherent systems.
\item Francesco Piazza, Francesco.Piazza@cnrs-orleans.fr, Max Planck Institute for the Physics of Complex Systems, physics, SIR models.
Has written one of the central papers modeling covid disease propagation from the SIR perspective. We have based our methods on their paper.
\item Ernesto Estrada, estrada@ifisc.uib-csic.es, Institute of Interdisciplinary Physics and Complex Systems , mathematics, graph theory, SIR models.
Has written a recent paper on the Gompertz model in relation to communicable disease modelling including close connections with graph theoretical concepts used in our paper. 
\item K. Rypdal, kristoffer.rypdal@uit.no, University of Tromsø, mathematics, SIR models.
Has written a paper showing that SIR models become Gompertz in the extreme of the Richard's parameter. We also follow this paper's arguments in our paper.
\item Michael Levitt (NAS Member), michael.levitt@stanford.edu, Stanford University, computational biology, Gompertz models.
Has a preprint on the gompertz model in relation to covid where they leave the explanation of why mortality follows gompertz as an open question.
\item Chris G. Antonopoulos, canton@essex.ac.uk , U. of Essex, physics, chaos and SIR models.
Has written on the SIR model and chaotic dynamics.
\item Gabor Vattay, vattay@elte.hu, Eötvös Loránd University Budapest, physics, SIR models
Has written a paper on SIR models showing that there are ``initial transients'' in the observations which are explained by us in the Gompertz model.
\item E. B. Postnikov, postnikov@kursksu.ru, Kursk State Univesity, physics, SIR models.
Has done work on SIR models using a mean field approach, but has a broach research interest that could be a benefit when reviewing our paper.
\item L. Kocarev, lkocarev@manu.edu.mk, Macedonian Academy of Sciences and Arts
, physics and mathematics, SIR models.
Has written papers on SIR models as they relate to non-markovian models.
\item Constantino Tsallis, tsallis@cbpf.br, Complexity Science Hub Vienna, Austria. Has done extensive work on long-range interaction and statistical mechanics. He might be able to see connections with his famous Tsallis entropy.


\end{enumerate} 

\end{document}
