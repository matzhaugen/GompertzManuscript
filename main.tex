\documentclass{article}
% \documentclass[review]{elsarticle}


\usepackage{arxiv}
\usepackage{booktabs}
\usepackage{siunitx}
\usepackage[utf8]{inputenc} % allow utf-8 input
\usepackage[T1]{fontenc}    % use 8-bit T1 fonts
\usepackage{hyperref}       % hyperlinks
\usepackage{url}            % simple URL typesetting
\usepackage{booktabs}       % professional-quality tables
\usepackage{amsfonts}       % blackboard math symbols
\usepackage{nicefrac}       % compact symbols for 1/2, etc.
\usepackage{microtype}      % microtypography
\usepackage{lipsum}   % Can be removed after putting your text content
\usepackage{graphicx}
\usepackage[square,numbers]{natbib}
\usepackage{doi}
\usepackage{amsmath}
\usepackage{float}

\newcommand{\insertPdfFig}[3]{
  \begin{figure}[H]
  \centering
  \includegraphics[width=8.4cm]{#1.pdf}
  \caption{#2}
  \label{fig:#1}
  \end{figure}
}

\newcommand{\insertSmallPdfFig}[3]{
  \begin{figure}[h]
  \centering
  \includegraphics[width=8.4cm]{#1.pdf}
  \caption{#2}
  \label{fig:#1}
  \end{figure}
}

% \title{On the nature of the Gompertz in relation to the epidemic in 2019-2021}
\title{Unifying Gompertzian growth with the communicable disease spreading paradigm}
\graphicspath{{figures/}}
%\date{September 9, 1985} % Here you can change the date presented in the paper title
%\date{}          % Or removing it

\author{Matz A. Haugen, \\
Independent Reseacher, Oslo\\
\texttt{matzhaugen@gmail.com} \\
  \And
  Dorothea Gilbert \\
  University of Oslo\\
  %% \AND
  %% Coauthor \\
  %% Affiliation \\
  %% Address \\
  %% \texttt{email} \\
  %% \And
  %% Coauthor \\
  %% Affiliation \\
  %% Address \\
  %% \texttt{email} \\
  %% \And
  %% Coauthor \\
  %% Affiliation \\
  %% Address \\
  %% \texttt{email} \\
}

% Uncomment to remove the date
%\date{}

% Uncomment to override  the `A preprint' in the header
%\renewcommand{\headeright}{Technical Report}
%\renewcommand{\undertitle}{Technical Report}
\renewcommand{\shorttitle}{On the nature of the Gompertz}

%%% Add PDF metadata to help others organize their library
%%% Once the PDF is generated, you can check the metadata with
%%% $ pdfinfo template.pdf
\hypersetup{
pdftitle={A template for the arxiv style},
pdfsubject={q-bio.NC, q-bio.QM},
pdfauthor={Matz A. Haugen},
pdfkeywords={},
}

\begin{document}
\maketitle

\begin{abstract}
Recently, a number of studies have shown that cumulative mortality followed a Gompertz curve in the initial Coronavirus epidemic period, March-April 2020. 
We show that the Gompertz curve is incompatible with the traditional communicable disease spreading hypothesis, and propose a new theory which better explains the nature of the mortality characteristics based on an environmental stressor. 
Second, we show that for the Gompertz curve to emerge, the stressor has to act on everyone simultaneously, rejecting the possibility of a disease propagation stage. 
Third, we show that the population acts like a coherent organism under growth/depletion. 
Finally, we connect the Susceptible-Infected-Recovered (SIR) model with our new theory and show that the SIR model is compatible with Gompertzian growth only when all nodes in the transmission network communicate with infinite speed and interaction.
\end{abstract}

% keywords can be removed
\keywords{Gompertz \and Coherence \and Covid \and Coronavirus \and Network Analysis \and Stochastic}

\section{Introduction}
Traditional communicable disease spreading theory assumes a pathogen which infects the population through a network of transmission. 
Following this line of reasoning it can be theoretically shown that in the early stages of an epidemic, growth follows a logistic-like curve, where the very beginning exhibits exponential growth, and for which analytic and semi-analytic solutions have recently emerged \citep{harko2014exact,kroger2020analytical,schlickeiser2021analytical,heng2020approximately}. 
A large body of research has emerged that explores how these models fit the recent mortality seen due to the Coronavirus epidemic \citep{carletti2020covid,cooper2020sir,postnikov2020estimation,munoz2021sir,cooper2022dynamical,saikia2021covid}. 

However, instead of showing logistic-like growth, observed cumulative mortality in the initial period March-April 2020 exhibits almost perfect resemblance to Gompertzian growth \citep{Gompertz1825} where the log-transformed cumulative mortality, or log-mortality for short, is exponentially \emph{decreasing} in time,
\begin{equation}
\label{eq:GompertzODE}
\frac{d}{dt}\ln{Y(t)} = -\beta\ln{\frac{Y(t)}{\tilde{Y}}} + \nu,
\end{equation} 
% Redundant definitions for full script
with constants $\tilde{Y}$, $\beta$, and $\nu$, and whose solution is given as
\begin{equation}
\label{eq:gomp_solution}
Y(t) = Y_\infty \left(\frac{Y_0}{Y_\infty}\right)^{e^{-\beta t}}
\end{equation}
$Y_{0}=Y(t = 0)$, $Y_{\infty}=Y(t\rightarrow \infty)=\tilde{Y}e^{\nu/\beta}$. 

% \insertSmallPdfFig{Gompertz_concept}{A conceptual display of the discrepancy seen between the logistic growth and the observations, superimposed by a Gompertz curve. Data is transformed to the straight line domain of the Gompertz curve for visual clarity, $f(Y)=\ln{\ln{1/Y}}$.}

This phenomenon is recorded by \citep{Ohnishi2020,Rypdal2020,Catala2020,rodrigues2020monte,Levitt2020}, showing Gompertz curves at national levels instead of the traditionally predicted logistic curves. What is causing such a discrepancy between reality and the current theory of communicable diseases \citep{castro2020turning}?

One could approach this conundrum from parametrization of the Susceptible-Infected-Recovered (SIR) model under a time-dependent infection/recovery rate ratio, $\phi(t)$. How would $\phi(t)$ have to behave? Start by recalling the SIR model for a pool of susceptible people of size $N$ evolving between the three states susceptible, $S(t)$, recovered, $R(t)$ and infected, $I(t)$,
\begin{align}
\label{eq:SIR}
\frac{dS}{dt}& = -\beta \frac{IS}{N}\\
\frac{dI}{dt}& = -\beta \frac{IS}{N} - \alpha I.\\
\frac{dR}{dt}& = \alpha I,
\end{align}
omitting the argument $t$ in each variable for brevity and where $\alpha$ and $\beta$ in this context signify recovery and infection rates.

A line of reasoning employed by \citet{Rypdal2020} to obtain the Gompertz curve is to linearize the SIR model by assuming
both the number of infected, $I(t)$, and cumulative infected, $Y(t)$, is much less than the total population, $N \gg Y \ge I$, which seems well founded at the national level,
\begin{align}
\label{eq:linSIR}
\frac{dY}{dt}& = \beta I\\
\frac{dI}{dt}& = (\beta - \alpha) I = \alpha (\phi(t) - 1) I,
\end{align}
and where the number of recovered, $R(t)$, is under this linearization decoupled from the other variables. 

They further assume the number of diseased is proportional to the number of cumulative infected, offset by a time lag, which allows us to use the same set of linearized equations to model the number of cumulative people diseased without loss of generality.

 Due to this linearization, the infection/recovery ratio, $\phi(t)$, will have to change as a function of time to accommodate for the boundary conditions. And since $I=I(Y)$, we combine (\ref{eq:linSIR}) via an instantaneous relative growth rate, $\gamma(t) = dY(t) /(Ydt) = \beta I / Y$, in turn parameterized by a scaling factor, $\theta$, representing the shape of the growth,
\begin{subequations}
\label{eq:rypdal}
\begin{align}
\frac{dY}{dt} =& \gamma(t) Y(t) \label{eq:rypdalODE}\\
\gamma(t) =& \frac{\gamma_{\infty}}{\theta}\left[1 - \left(\frac{Y}{Y_{\infty}}\right)^{\theta} \right] \label{eq:rypdalGamma},
\end{align}
\end{subequations}
where $\gamma_{\infty} = \gamma(t\rightarrow \infty)$. This parametrization is the commonly used Richard's growth curve \citep{richards1959flexible}, also called $\theta$-logistic growth, and has been used by others \citep{wu2020generalized}. Although not immediately justified in the communicable disease theory, one could imagine that $\theta$ represents non-linear network behavior \citep{petroni2020logistic}. Note that at $\theta=1$, the traditional logistic growth curve is obtained, while at $\theta\rightarrow \infty$ we recover the exponential (Malthusian) explosion.

The observed Gompertzian mortality curves are realized in the limit $\theta \rightarrow 0$, with the relative growth rate, 
\begin{equation}
\label{eq:rypdalLimit}
\gamma(t) = \lim_{\theta \rightarrow 0}\frac{\gamma_{\infty}}{\theta}\left[1 - \left(\frac{Y}{Y_{\infty}}\right)^{\theta} \right]
= \gamma_{\infty}\ln{\frac{Y_{\infty}}{Y}}
\end{equation}

 At this limit the growth rate approaches infinity as $Y \rightarrow 0$, which seems odd under the hypothesis that the pathogen has just started spreading. 
 The Gompertzian limit also implies a decreasing relative growth rate from the very first time point, which under the SIR model seems unlikely given the large pool of susceptible people in the beginning. 
 One would rather expect a near-constant relative growth rate in the beginning due to a disease propagation stage. 
 \citet{Rypdal2020} suggest that the decreasing relative growth rate is caused by social and political mitigating efforts, but these hardly justify such coherent and consistent mortality characteristics across countries. 
 
 Perhaps a more likely scenario from which a Gompertz curve would emerge is the selective infection of central nodes in the transmission network resulting in an immediate decrease in relative growth. 
 On the other hand, an infection of peripheral nodes should cause immediate exponential growth. 
 \citet{herrmann2020covid} studied a networked SIR model on a variety of networks, but did not elaborate on a possible fit to a Gompertz curve. 
 Although it may be possible to realize Gompertzian growth from a special network, firm theoretical work has yet to be done to show this connection. 
 We will touch upon how the Gompertz curve emerges from one such network below, under some caveats.

Without the linearization and the somewhat arbitrary $\theta$-parametrization, one can still obtain from the communicable disease models growth which is almost Gompertzian, i.e. a straight line under a double-log transform of (\ref{eq:gomp_solution}), viz.
\begin{equation}
\label{eq:GOMP_D}
\ln{(\ln{(Y_{\infty}/Y)})} = -\beta t + k,
\end{equation}
with $k=\ln{\ln{\frac{Y_{\infty}}{Y_{0}}}}$. To show this, we follow \citet{carletti2020covid} and consider the extended version of the SIR the model by including a group of diseased, $D(t)$, such that $N = S(t) + I(t) + R(t) + D(t)$, also called the SIRD model. This requires an addition to our original set of equations (\ref{eq:SIR}), with a growth equation of the diseased group at some rate $\delta$ relative to the current infected group,
\begin{equation}
\frac{dD}{dt} = \delta I.
\end{equation}

Some algebraic manipulations reveal that the diseased group is described by a single equation, viz.
\begin{equation}
\label{eq:SIRD_D}
\frac{dD}{dt} = \eta( 1 - e^{-\xi D}) - \kappa D,
\end{equation}
where $\eta = \delta N$, $\xi = \beta/(\delta N)$, $\kappa = \delta + \alpha$, and where $N$ is assumed synonymous to the initial susceptible pool of people. This pool of people cannot be obtained from deaths alone, but can be inferred by assuming a known ratio between mortality and recovery, $\delta$ and $\alpha$. 
When only modeling the diseased however, knowledge of $N$ is irrelevant and the three parameters in (\ref{eq:SIRD_D}) are sufficient.

Note here that at small values of D follow logistic growth,
\begin{equation}
\frac{dD}{dt} = b D (1 - D/K),
\end{equation}
obtained with a second order expansion of the exponential term in \ref{eq:SIRD_D} and with $b=\eta\xi - \kappa$ and $K=2b/\eta\xi^2$. The SIR model exhibits similar properties and its discrepancy with initial observations has been noted by others \citep{vattay2020forecasting}.

With the full 3-parameter single ODE of the diseased group in the SIRD model, one can obtain a curve quite close to a straight line in the double-log domain even in the initial observations, although there will always be a non-zero concavity. 
Meanwhile, the Gompertz growth model can explain the initial observations with only two parameters obtainable through linear regression, while the third parameter accounts for the discrepancy between the final observation, $Y_{max}$, and $Y_{\infty}$, an effect which is manifested as a tapering of the initial straight line in the double-log domain. 
Thus, through linear regression estimates, the Gompertz model avoids the possibility of non-identifiability issues and unstable optima \citep{roda2020difficult}.
Comparisons between the two models and the inevitable concavity of the SIRD model can be seen when fitting observations of diseased attributed to the Coronavirus in a variety of countries (Fig. \ref{fig:Gompertz_vs_SIRD_lnln_ymax} and Supplemental Information). 

Note also that the SIRD model above considers average macroscopic behavior of an ensemble of microscopic units justified through mean-field theory \citep{smilkov2014beyond}. However, this model does not consider network effects explicitly. Rather, all entities are connected and communicating instantaneously as shown by \citet{mombach2002mean}.
However, even with such strong assumptions, it is odd that observed mortality never exhibits a stage of near-exponential growth as this macroscopic SIRD model predicts, but rather a constant negative slope in the double-log domain. 
We are thus prompted to look for another model which can explain the observed mortality patterns.

\insertPdfFig{Gompertz_vs_SIRD_lnln_ymax}{Cumulative number of diseased, transformed by $g(Y)=\ln{\ln{({Y_{max} / Y(t))}}}$ plotted against number of days elapsed after $Y(t) / Y_{max}>0.005$, comparing a SIRD with a Gompertz model. Both models are fit using non-linear least squares according to equations (\ref{eq:GompertzODE}) and (\ref{eq:SIRD_D}) in the text after normalizing, although the Gompertz curve can also be obtained with a simple linear fit using (\ref{eq:GOMP_D}). The temporal evolution of the SIRD model is obtained from the Runge-Kutta algorithm. Each plot is annotated with an inferred transmission rate in the SIRD model, assuming bi-normally distributed parameter estimates. Fitting is done with Python-Scipy and a 6-day windowed average of deaths. Observations are taken from the Github repository compiled by the Center for Systems Science and Engineering (CSSE) at Johns Hopkins University, Baltimore, USA \citep{dong2020interactive}.} 

\section{An alternative theory for observed mortality}
An alternative line of reasoning that does not rely on the framework of communicable diseases is that the biosphere was perturbed by an external stressor, initiating a stress response to eventually bring mortality rates back to stability. 
This could explain the immediate dampening of mortality growth observed. 
It could also explain the lack of correlation between population density and mortality (or infection) rates observed by some \citep{Hamidi2020,Hamidi2020a,Carozzi2020,Arpino2020,khavarian2021high,barak2021urban}.

Inspired by  \citet{de2014stochastic}, the stressor can be modeled as a multiplicative stochastic dampening term at the along with the countering force of the immune system. 
As with all multiplicative processes, it is convenient to log-transformed mortality with dimensionless variables $F(t)=Y(t)/Y_\infty$ and $Z(t)=\ln{F}(t)$, from which a natural perturbation model emerges,
\begin{equation}
\label{eq:microscopic}
dZ(t)= -\beta Z(t) dt + \sqrt{\sigma}dW(t),
\end{equation}
where $dW(t)$ is a delta-correlated Wiener process with zero mean, making $Z(t)$ a stochastic process. This perturbation model is called an Ornstein–Uhlenbeck process \citep{risken1996fokker}. It is also interpreted as Newton's equation of motion with friction and a random force (Langevin's equation), and a continuous version of an Auto-Regressive(1) model \citep{akaike1970statistical}. The diffusion coefficient, $\sigma$, represents the strength of the perturbation, which we directly see if we recast this equation in terms of mortality while introducing a new parameter $K=\exp{(-\frac{\sigma}{2\beta})}$, viz. 
\begin{equation}
\label{eq:microscopicOriginal}
dF(t) = \left\{\frac{\sigma}{2}F(t) - \beta F(t)\ln\left[\frac{F(t)}{K}\right]\right\}dt + \sqrt{\sigma}F(t)dW(t).
\end{equation}

The first term on the right hand side represents the growth due to the stressor, while the second represents the stress response. 

To obtain the deterministic observable, we first take the average in the log-domain (\ref{eq:microscopic}) and transform back to the original domain, 
\begin{equation}
d\langle\ln{F(t)}\rangle = -\beta \langle\ln{F(t)}\rangle dt.
\end{equation}
Then use the property that the average of log-quantities is the logarithm of the median quantity, where we denote the median of $F(t)$ as $m(t)$, 
\begin{equation}
\label{eq:medianGomp}
d\ln{m(t)} = -\beta \ln{m(t)} dt,
\end{equation}
which corresponds with the familiar deterministic Gompertz differential equation in (\ref{eq:GompertzODE}) with $Y(t) = Y_\infty m(t)$. 
Comparing the stochastic stressor $\sigma$ in \ref{eq:microscopicOriginal} with the deterministic growth equation in (\ref{eq:GompertzODE}) gives $\nu=\sigma/2$, suggesting that the stressor is indeed the source of growth, while $\beta$ is the growth-limiting factor. 
We also see that by comparing (\ref{eq:rypdalODE}) and (\ref{eq:rypdalLimit}) with the stochastic counterpart in (\ref{eq:microscopicOriginal}) that the final growth level is governed by the stressor magnitude,
\begin{equation}
\sigma = 2\gamma_{\infty}\ln{(Y_{\infty})}.
\end{equation}

Thus, a more direct interpretation of the observations is that mortality was caused by a planetary perturbation, modeled as a random process, to which organisms gradually develop resistance at a geometric rate \citep{boxenbaum2017hypotheses,neafsey1988gompertz}. 
Here, a geometric rate is an exponential rate in the log-transformed domain, which is the natural transformation for many processes in nature \citep{zhang1994log}. 
The log-normal distribution of the abundance of $F(t)$ can be obtained by directly solving the stochastic equation in \ref{eq:microscopicOriginal} \citep{skiadas2010exact,petroni2020gompertz}, or from thermodynamic principles \citep{sitaram1984statistical,gunasekaran1982lon}. 
Intuitively, this is seen by noting that the solution to the perturbation model in the log-domain (\ref{eq:microscopic}) is Gaussian in the variable $Z(t)$, thus suggesting a log-normal distribution of $F(t)$. 

\section{Unifying the SIR model and the Gompertz model}
The remarkable observation that the log-transformed domain is the natural one merits closer study. First, juxtapose the logistic model with the Gompertz model,

\begin{align}
\label{eq:compare}
\begin{split}
\dot{F} = \frac{d}{dt}F(t) = \beta F(t) (1-F(t)) \quad&\quad \text{Logistic}\\ 
\frac{d}{dt}\ln{(F(t))} = -\beta \ln{(F(t)}) &\quad \text{Gompertz},
\end{split}
\end{align}
where $F$ is here deterministic. 

In the logistic model, we recognize the rightmost side of the logistic equation as the transmission term in an SIR model, but also as a linear interaction term between the two macroscopic states. 
As mentioned earlier, this procedure is a mean field approximation with an implied average interaction between the variables. 
Thus, all dynamics are governed by macroscopic deterministic variables parametrized by a transmission rate.

A microscopic solution could be modeled by splitting the system into $N$ microscopic deterministic units, $F(t) \rightarrow f_1(t), f_2(t),$ etc, from which macroscopic growth would be obtained by taking the average over the individual units,
\begin{equation}
F(t) = \frac{1}{N}\sum_i^N f_i(t),
\end{equation}
where the lower case $f_i$ emphasize the microscopic quality of the variables, and is here interpreted as probability of infection.

One could add network interaction necessitating a corresponding matrix version of (\ref{eq:compare})
\begin{equation}
\label{eq:networkSIR}
\frac{d f_i}{dt} = \beta (1-f_i)\sum_j{a_{ij}}f_j \quad \forall i,
\end{equation}
using the shorthanded $f_i=f_i(t)$ and with a fixed correlation governed by the network's growth rate, $b$, and adjacency matrix, $\{a_{i,j}\}=\mathbf{A} \in \mathbb{R}^{N \times N}$, a binary matrix with ones where the $i^{th}$ and $j^{th}$ nodes are connected, and zeroes otherwise. Notice here that there is an implied causality going from the infected to susceptible, which will become relevant below. Furthermore, a linear correlation between variables is seen as the partial derivatives of the instantaneous growth rate with respect to pairs of microscopic variables,
\begin{equation}
\frac{\partial^2 \dot{F}}{\partial f_i \partial f_j} = -\frac{\beta}{N} a_{i,j}.
\end{equation}
Still, no Gompertz curve will emerge at the onset of the growth process. \citet{estrada2022networked} provide illuminating analysis on this topic.

In contrast, as shown in (\ref{eq:microscopic}), the Gompertz model is implied by a multiplicative stochastic perturbation and thus suggests correlated mortality growth, viz.
\begin{equation}
F(t) = \exp{\left[\frac{1}{N}\sum_i^N z_i(t)\right]} = \left[\prod_i^N f_i(t)\right]^{1/N},
\end{equation}
where the rightmost equality is due to the central limit theorem where $z_i$ is a random variable as in \ref{eq:microscopic} while $f_i$ is here deterministic. 
Under this model, correlation between entities is present at all orders in the original domain\footnote{It is illuminating to at this point compare with Gompertz' Law of Mortality, $\frac{d}{dt}\ln{(1 - F(t))} = -b$, for $t$ more than ~25 years, which yields a naturally uncorrelated macroscopic curve $1 - F(t) = \exp{(-bt)}$ \citep{shklovskii2005simple}. 
In our context, the uncorrelated feature emerges since the force of mortality is not a function of the growth itself, as we see in the text, but rather of time.}.

Thus, the emergence of the Gompertz curve at the macroscopic level suggests that the system is correlated, or coherent, presumably as a result of the simultaneous exposure to the same underlying stressor, but also due to the implied log-normal nature of the microscopic entities \citep{zhang1994log}.
We can now further appreciate Richard's parametrization as a transition from non-collaborative to collaborative growth as $\theta \rightarrow 0$. 
This feature of $\theta$ was also obtained by \citet{petroni2020logistic} by interpreting the $\theta-$logistic growth rate in (\ref{eq:rypdal}) as non-linear resource availability dependent on the overall magnitude, $Y(t)$, with growth at $\theta \rightarrow 0$ named ``maximally coherent''. 
\citet{molski2003coherent} made a similar observation that Gompertzian growth is the coherent state in a quantum mechanical system with a time-dependent potential, an interpretation which sheds further light on the temporal nature of the postulated stress response.
This quantum mechanical system has also been used to describe coherent energy states of diatomic molecules in space \citep{morse1929diatomic}. 
In the field of quantum physics, \emph{coherence} is a well-defined mathematical property first explored by \citet{glauber1963coherent} in the context of electromagnetic fields.

\subsection{Unification through a modified SIR model}
However, it is possible to reconcile the SIR model with the Gompertz curve from yet another argument. 
Inspired by the observation that in the linearized approximation there are only two coupled states, infected and susceptible, we augment the interaction term to higher orders. 
Then, we reverse the causality where the population of infected are now dependent of the population of the susceptible instead of the other way around as in (\ref{eq:networkSIR}),
\begin{equation}
\label{eq:reverseNetworkSIR}
\frac{d f_i}{dt} = \beta f_i\sum_j{a_{ij}}(1-f_j) \quad \forall i.
\end{equation} 
This crucial change is based on the environmental stressor hypothesis rather than a communicable disease assumption.
Furthermore, we will for the sake of simplicity assume all nodes in the network have exactly one neighbor and that there exists a unique path between all nodes,
\begin{equation}
\sum_i a_{ij} = 1 \quad \forall j.
\end{equation} 
If we let $s_i = 1 - f_i$ be the probability of being susceptible, the augmented and causality-reversed SIR model becomes 
\begin{equation}
\frac{d f_i}{dt} = \beta f_i\sum_j{a_{ij}}(s_{j} + s^2_j/2 + ...) - \alpha f_i\quad \forall i,
\end{equation}

Now use the Taylor series $x+x^2/2+... = -\ln(1-x)$, viz.

\begin{equation}
\frac{d f_i}{dt} = \beta f_i\sum_j{a_{ij}}\ln{(1 - s_{j})} - \alpha = \beta f_i\sum_j{a_{ij}}\ln{f_{j}} - \alpha f_i \quad \forall i,
\end{equation}
As we are interested in the aggregate macroscopic behavior, we take averages in the log domain, exploit our setup where $\mathbf{A}$ is a single mapping from one node to another, and simplify to
\begin{equation}
\frac{d}{{dt}} \ln\left[\prod_i{f_i}\right]^{\frac{1}{N}} = -\beta\ln\left[\prod_i{f_i}\right]^{\frac{1}{N}} - \alpha.
\end{equation}
If $\alpha\rightarrow 0$, then this equation will exhibit Gompertz growth in the geometric ensemble average of microscopic units. The interpretation of $\alpha\rightarrow 0$ could be that the limiting factor emerges purely from the growth rate without the need for a second growth-limiting parameter, in line with the previous crucial hypothesis of causality reversal stated above. One further simplification could be seen in equating the logarithm of the geometric mean with the logarithm of the median of the set of $f_i$ to obtain \ref{eq:medianGomp},
\begin{equation}
\frac{d}{dt}\ln{m(t)} = -\beta \ln{m(t)}.
\end{equation}
Finally, in this exposition, all the nodes in the network are communicating instantaneously. 

\section{Conclusion}
In conclusion, we have shown that Gompertz growth follows from infinite interactions between the susceptible and infected states, and the perceived pathogen travels at infinite speed throughout the population, rejecting the possibility of a disease propagation stage through a perceived transmission network. In this vein, Richard's parameter $\theta$ in his growth model paradigm can be related to the number of higher order interactions with the susceptible and the infected in an SIR model \citep{richards1959flexible}, where infinite interactions corresponds to $\theta\rightarrow 0$. 

We further show that the observed mortality across countries can be explained by a model where the biological system is stressed by a ubiquitous and simultaneous stressor eliciting a corresponding stress response through which gradual return to pre-epidemic conditions are mediated. The stressor is modeled as a stochastic perturbation in the log-transformed domain of effects where correlation between people or microscopic entities is present at all orders. From this model, we draw parallels between the coherent behavior of the population's mortality evolution during an epidemic and the spatial coherence of quantum mechanical systems, borrowing the definition of coherence from quantum physics \citep{molski2003coherent}.

Thus, we see growth on a spectrum: In one extreme we find non-collaborative growth models or models with parameterized linear interaction effects, and in the other extreme we see a field of microscopic entities coherently sharing information much like quantum entangled particles. The emergence of coherent quantum phenomena at the macroscopic level suggests that no longer can the microscopic world claim a monopoly on quantum physics, especially as it relates to biology. One might be surprised to find that temporal evolution of human mortality during epidemics can behave like the spatial energy distribution of quantum coherent systems.


\newpage
\section*{Supplement}

\insertPdfFig{Gompertz_vs_SIRD}{Cumulative diseased plotted against number of days after $Y(t) / Y_{max}>0.005$, comparing an SIRD with a Gompertz model. Both models are fit using non-linear least squares according to equations (\ref{eq:GompertzODE}) and (\ref{eq:SIRD_D}) in the text. Fitting is done with Python-Scipy and a 6-day windowed average of deaths. Observations are taken from the Github repository compiled by the Center for Systems Science and Engineering (CSSE) at Johns Hopkins University, Baltimore, USA \citep{dong2020interactive}.} 


\begin{table}[H]
\centering
\caption{Estimated values for the SIRD model fitted to various countries' observations normalized by the last observation, $Y_{max}$, as shown in Figure \ref{fig:Gompertz_vs_SIRD_lnln_ymax} using notation from \ref{eq:SIRD_D}. Note that due to normalization values will be different from \citet{carletti2020covid}.}
\begin{tabular}{llll}
\toprule
{} &           $\eta\times 10$ &      $\xi$ &            $\kappa\times 10^{2}$ \\
\midrule
Belgium        &    1.0(1) &  2.9(3) &   9.3(1) \\
Portugal       &   0.55(4) &  4.2(4) &   4.5(4) \\
Austria        &    1.3(3) &  2.3(4) &    11(2) \\
Switzerland    &   1.24(1) &  2.4(2) &  11.0(7) \\
Sweden         &    0.5(2) &    3(1) &     3(2) \\
Italy          &   0.43(1) &  5.6(2) &   3.7(1) \\
France         &   0.88(7) &  3.4(3) &   8.1(6) \\
Spain          &  0.525(1) &  6.2(2) &   4.8(1) \\
Germany        &   0.78(8) &  3.0(3) &   6.5(7) \\
Denmark        &   0.50(4) &  5.2(6) &   4.2(4) \\
United Kingdom &   0.54(2) &  4.7(2) &   4.3(2) \\
Netherlands    &   0.51(3) &  5.0(4) &   4.3(3) \\
Turkey         &   0.84(9) &  2.8(3) &   6.9(8) \\
Iran           &  0.322(7) &  7.2(3) &  2.51(9) \\
US             &  0.339(8) &  7.0(3) &   1.8(1) \\
China          &    3.8(9) &  1.2(2) &    26(4) \\
\bottomrule
\end{tabular}

\end{table}


\begin{table}[H]
\centering
\caption{Estimated values for the Gompertz model fitted to various countries' observations normalized by the last observation, $Y_{max}$, as shown in Figure \ref{fig:Gompertz_vs_SIRD_lnln_ymax} using notation from \ref{eq:GompertzODE}.}
\begin{tabular}{llll}
\toprule
{} &             $\nu\times 10$ &           $\beta\times 10^{2}$ &        $\tilde{Y}\times 10^{2}$ \\
\midrule
Belgium        &   2.74(1) &   8.1(2) &  3.6(3) \\
Portugal       &  2.242(6) &  6.23(9) &  3.1(1) \\
Austria        &   2.74(1) &   8.1(2) &  3.6(2) \\
Switzerland    &  2.740(7) &   8.1(1) &  3.6(2) \\
Sweden         &   1.73(3) &   4.3(3) &  2.6(4) \\
Italy          &  2.107(6) &  5.88(9) &  3.0(1) \\
France         &  2.752(1) &   8.2(2) &  3.6(2) \\
Spain          &  2.556(8) &   7.5(1) &  3.4(2) \\
Germany        &  2.335(7) &   6.5(1) &  3.2(1) \\
Denmark        &   2.32(1) &   6.6(2) &  3.2(3) \\
United Kingdom &  2.333(6) &  6.54(9) &  3.2(1) \\
Netherlands    &  2.311(7) &   6.5(1) &  3.2(1) \\
Turkey         &  2.306(7) &   6.4(1) &  3.2(1) \\
Iran           &  1.862(7) &  5.04(9) &  2.7(1) \\
US             &  2.014(8) &   5.3(1) &  2.9(1) \\
China          &  3.155(8) &   9.4(2) &  3.6(2) \\
\bottomrule
\end{tabular}

\end{table}
\bibliographystyle{plainnat}
\bibliography{../references2}  %%% Uncomment this line and comment out the ``thebibliography'' section below to use the external .bib file (using bibtex) .


%%% Uncomment this section and comment out the \bibliography{references} line above to use inline references.
% \begin{thebibliography}{1}

%   \bibitem{kour2014real}
%   George Kour and Raid Saabne.
%   \newblock Real-time segmentation of on-line handwritten arabic script.
%   \newblock In {\em Frontiers in Handwriting Recognition (ICFHR), 2014 14th
%       International Conference on}, pages 417--422. IEEE, 2014.

%   \bibitem{kour2014fast}
%   George Kour and Raid Saabne.
%   \newblock Fast classification of handwritten on-line arabic characters.
%   \newblock In {\em Soft Computing and Pattern Recognition (SoCPaR), 2014 6th
%       International Conference of}, pages 312--318. IEEE, 2014.

%   \bibitem{hadash2018estimate}
%   Guy Hadash, Einat Kermany, Boaz Carmeli, Ofer Lavi, George Kour, and Alon
%   Jacovi.
%   \newblock Estimate and replace: A novel approach to integrating deep neural
%   networks with existing applications.
%   \newblock {\em arXiv preprint arXiv:1804.09028}, 2018.

% \end{thebibliography}


\end{document}